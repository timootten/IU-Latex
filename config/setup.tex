\usepackage[ngerman]{babel} % neue deutsche Trennungsregeln, etc
\usepackage[T1]{fontenc} % enable hyphenation for languages with accented characters
\usepackage{graphicx}
\usepackage{fancyhdr}
\usepackage{titlesec}
\usepackage{xurl}
\usepackage[breaklinks=true]{hyperref}
\usepackage{setspace} % Abstände zwischen Absätzen
\setstretch{1.5} % 1,5 Zeilenabstand
\usepackage[left=2cm, right=2cm, top=2cm, bottom=2cm]{geometry} % Seitenränder
\usepackage{setspace} % For line spacing

\usepackage{fontspec}

% Calibri explizit als Sans-Serif-Font definieren
\setsansfont[
  Path =./assets/,
  BoldFont = calibri-bold.ttf,    % Pfad zur fetten Version
  ItalicFont = calibri-italic.ttf,  % Pfad zur kursiven Version
  BoldItalicFont = calibri-bold-italic.ttf % Pfad zur fett-kursiven Version
]{Calibri-regular.ttf}

\renewcommand{\familydefault}{\sfdefault} % Set default font to Calibri


\usepackage{tocloft}

% Schriftart für Inhaltsverzeichnis auf Calibri setzen
\renewcommand{\cftsecfont}{\normalfont\fontsize{11pt}{1.5}\selectfont} % 1. Ebene
\renewcommand{\cftsubsecfont}{\normalfont\fontsize{11pt}{1.5}\selectfont} % 2. Ebene
\renewcommand{\cftsubsubsecfont}{\normalfont\fontsize{11pt}{1.5}\selectfont} % 3. Ebene

% Nur die erste Ebene fett
\renewcommand{\cftsecfont}{\normalfont\bfseries\fontsize{11pt}{1.5}\selectfont} % 1. Ebene fett
\renewcommand{\cftsubsecfont}{\normalfont\fontsize{11pt}{1.5}\selectfont} % 2. Ebene normal
\renewcommand{\cftsubsubsecfont}{\normalfont\fontsize{11pt}{1.5}\selectfont} % 3. Ebene normal

% Linksbündige Ausrichtung
\renewcommand{\cfttoctitlefont}{\normalfont\bfseries\Large\raggedright} % Titel "Inhaltsverzeichnis"
\renewcommand{\cftsecpagefont}{\normalfont} % Seitenzahlen normal

\renewcommand{\cftdot}{.} % Setzt die Punkte im Inhaltsverzeichnis
\renewcommand{\cftsecdotsep}{\cftdotsep} % Falls du "section" Punkte setzen willst
\renewcommand{\cftsubsecdotsep}{\cftdotsep} % Falls du "subsection" Punkte setzen willst
\renewcommand{\footnotesize}{\fontsize{10pt}{1.5pt}\selectfont}

% automatischens Einrücken von Absätzen verhindern
\usepackage{changepage}
\setlength{\parindent}{0pt}
% 6pt Abstand nur zwischen Absätzen
\setlength{\parskip}{6pt}{}

% Bibliographie & Sondereinstellungen
\usepackage[babel]{csquotes}
\usepackage[backend=biber, style=apa, pagetracker, apamaxprtauth=20 ]{biblatex}
\makeatletter
\renewcommand*{\apablx@ifrevnameappcomma}{\@secondoftwo}
\makeatother
\DefineBibliographyExtras{ngerman}{%
  \renewcommand*{\finalandcomma}{}%
}


%% Bibliographie einbinden
\addbibresource{references.bib} % your bib file

% Blocksatz und Silbentrennung aktivieren
\sloppy

% Überschriftenformatierung
\titleformat{\section}
  {\fontsize{16pt}{1.5}\bfseries\raggedright} % 1. Ebene: 16pt, linksbündig
  {\thesection}{1em}{}

\titleformat{\subsection}
  {\fontsize{14pt}{1.5}\bfseries\raggedright} % 2. Ebene: 14pt, linksbündig
  {\thesubsection}{1em}{}

\titleformat{\subsubsection}
  {\fontsize{11pt}{1.5}\bfseries\raggedright} % 3. Ebene: 11pt, linksbündig
  {\thesubsubsection}{1em}{}

% Abstände vor und nach Überschriften
\titlespacing*{\section}{0pt}{12pt}{12pt}
\titlespacing*{\subsection}{0pt}{12pt}{6pt}
\titlespacing*{\subsubsection}{0pt}{12pt}{6pt}